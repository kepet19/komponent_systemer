\section{Javalab}
I denne opgave skulle man lave et ekstra komponent som kunne loades ind via
java service loader. Java service loader er en implementering af Whiteboard
Component Model. I denne opgave lærer man hvordan man registere en klasse som
kan blive loadet ind via java service loader.


\subsection{Register komponent}
først lave jeg en ny "Enemy" komponent som er baseret på player, da mange af
tingende er de samme. 

Derefter tilføjede jeg Enemy til \code{rodens/root} og \code{./pom.xml} filen,
hvor den kom til at stå som et ekstra modul så maven vidste den skulle compiles
med. Derefter tilføjede jeg "Enemy" til dependencies inde i "Core/pom.xml". så
java kunne finde de interfaces som vi ledte efter.  En af de key items er at
man skal have en "resource" mappe med hvor er er en "META-INF" mappe og i den
mappe har man en fil med navnet på det interface man siger man implementere.
inde i filen siger man hvilke klasse der implementere dette interface.

Her har jeg en træ structur af hvordan filerne ser ud.

\hfill \linebreak \hline 
\dirtree{%
.1 Enemy.
.2 pom.xml.
.2 src.
.3 main.
.4 java.
.5 dk.sdu.mmmi.cbse.enemysystem.
.6 Enemy.java.
.6 EnemyControlSystem.java.
.6 EnemyPlugin.java.
.3 resources.
.4 META-INF.
.5 navent på det interface man siger man implementere.
.5 dk.sdu.mmmi.cbse.common.services.IEntityProcessingService.
.5 dk.sdu.mmmi.cbse.common.services.IGamePluginService.
}
\hline \hfill \linebreak

\begin{lstlisting}[caption={dk.sdu.mmmi.cbse.common.services.IEntityProcessingService}, label={lst:App}, language=java]
dk.sdu.mmmi.cbse.enemysystem.EnemyControlSystem
\end{lstlisting}
""
\begin{lstlisting}[caption={dk.sdu.mmmi.cbse.common.services.igamepluginservice}, label={lst:app}, language=java]
dk.sdu.mmmi.cbse.enemysystem.enemyplugin
\end{lstlisting}


\subsection{Loade modulerne ind via java ServiceLoader}
I klassen Game som ligger i modulet \code{Core} bruger klassen der hedder
\code{SPILocator} som ligger I modulet \code{Common}. \code{SPILocator} har en
methode til at loade klasser ind med et bestemt interface. \code{SPILocator}
implmentering kan ses i \textbf{Listing \ref{lst:SPILocator.java}}.
\code{SPILocator} bruger java standard service loader som kan ses på linje 20
I \textbf{Listing \ref{lst:SPILocator.java}} til at loade de klasser som
implementere det interface man har givet til serviceloaderen. 


Java service loader søger efter et bestemt interface I jar filerne. Jar filerne
inholder en mappe som hedder \code{META-INF/} hvor de så har en under mappe der
hedder \code{services/}. I den mappe søger java service loader så efter et
filnavn med det fulde interface navn(pakken er også med). 


\begin{lstlisting}[caption={dk.sdu.mmmi.cbse.common.util.SPILocator.java}, label={lst:SPILocator.java}, language=java]
package dk.sdu.mmmi.cbse.common.util;

import java.util.*;

public class SPILocator {

    @SuppressWarnings("rawtypes")
    private static final Map<Class, ServiceLoader> loadermap = new HashMap<Class, ServiceLoader>();

    private SPILocator() {
    }

    @SuppressWarnings("unchecked")
    public static <T> List<T> locateAll(Class<T> service) {
        ServiceLoader<T> loader = loadermap.get(service);

        boolean printStatement = false;

        if (loader == null) {
            loader = ServiceLoader.load(service);
            loadermap.put(service, loader);
            printStatement = true;
        }

        List<T> list = new ArrayList<T>();

        if (loader != null) {
            try {
                for (T instance : loader) {
                    list.add(instance);
                }
            } catch (ServiceConfigurationError serviceError) {
                serviceError.printStackTrace();
            }
        }

        if (printStatement) {
            System.out.println("Found " + list.size() + " implementations for interface: " + service.getName());
        }

        return list;
    }
}
\end{lstlisting}


