\section{OSGiLab}
I OSGI lab skal der laves 2 ny moduler, fordi der skal laves en med OSGi
Declarative Services(Dependency injection) og en med OSGi BundeContext
API(Whiteboard komponent modellen).


\subsection{OSGi Declarative Services}
I OSGi declarative services bruger vi en "osgi.bnd" til at fortælle OSGi hvilke
activationpolicy der skal værer. Den fortæller også noget om hvor den kan finde
noget om de service der bliver leveret fra modulet se \textbf{Listing
\ref{lst:osgi.bnd}}.

\hfill \linebreak \hline 
\dirtree{%
.1 Player.
.2 \textbf{osgi.bnd}.
.2 pom.xml.
.2 src.
.3 main.
.4 java.
.5 dk.sdu.mmmi.cbse.playersystem.
.6 Player.java.
.6 PlayerControlSystem.java.
.6 PlayerPlugin.java.
.3 resources.
.4 META-INF.
.5 entityprocessor.xml.
.5 \textbf{gameplugin.xml}.
}
\hline \hfill \linebreak

\begin{lstlisting}[caption={osgi.bnd}, label={lst:osgi.bnd}, language=java]
Bundle-SymbolicName: Player
Bundle-ActivationPolicy: lazy
Service-Component: META-INF/entityprocessor.xml, META-INF/gameplugin.xml
\end{lstlisting}

I \textbf{Listing \ref{lst:gameplugin.xml}} ses på linje 3 at det er
implementation klassen som den peger på og den implementere interfaces som er
listest på linje 5.

\begin{lstlisting}[caption={gameplugin.xml}, label={lst:gameplugin.xml}, language=xml]
<?xml version="1.0" encoding="UTF-8"?>
<scr:component xmlns:scr="http://www.osgi.org/xmlns/scr/v1.1.0" name="dk.sdu.mmmi.cbse.playersystem.plugin">
    <implementation class="dk.sdu.mmmi.cbse.playersystem.PlayerPlugin"/>
    <service>
        <provide interface="dk.sdu.mmmi.cbse.common.services.IGamePluginService"/>
    </service>
</scr:component>
\end{lstlisting}



% #### =========== bundle context API =========== ####
\newpage
\subsection{OSGI BundleContext API}

\hfill \linebreak \hline 
\dirtree{%
.1 Enemy2.
.2 \textbf{pom.xml}.
.2 src.
.3 main.
.4 java.
.5 dk.sdu.mmmi.cbse.osgienemy2.
.6 \textbf{Activator.java}.
.6 Enemy2.java.
.6 Enemy2Plugin.java.
.6 Enemy2Processor.java.
}
\hline \hfill \linebreak

\subsection{Register komponent}

I \textbf{Listing \ref{lst:pom-activator}} ses den pom.xml som peger på den
klasse som har implementeret bundle activator.

\begin{lstlisting}[caption={pom.xml}, label={lst:pom-activator}, language=xml]
    <build>
        <plugins>
            <plugin>
                <groupId>org.apache.felix</groupId>
                <artifactId>maven-bundle-plugin</artifactId>
                <extensions>true</extensions>
                <configuration>
                    <instructions>
                        <Bundle-Activator>dk.sdu.mmmi.cbse.osgienemy2.Activator</Bundle-Activator>
                    </instructions>
                </configuration>
            </plugin>
        </plugins>
    </build>
\end{lstlisting}


I \textbf{Listing \ref{lst:Activator.java}} ses den klasse som har implementeret
bundle activator.

\begin{lstlisting}[caption={Activator.java}, label={lst:Activator.java}, language=java]
package dk.sdu.mmmi.cbse.osgienemy2;

import dk.sdu.mmmi.cbse.common.services.IEntityProcessingService;
import dk.sdu.mmmi.cbse.common.services.IGamePluginService;
import org.osgi.framework.BundleActivator;
import org.osgi.framework.BundleContext;

public class Activator implements BundleActivator {

    @Override
    public void start(BundleContext context) throws Exception {

        context.registerService(IGamePluginService.class, new Enemy2Plugin(), null);
        context.registerService(IEntityProcessingService.class, new Enemy2Processor(), null);
    }

    public void stop(BundleContext context) throws Exception {
    }

}
\end{lstlisting}





\subsection{Komponentdet bliver loadet og unloadet via Apache gogo shell}
Se video \href{https://www.youtube.com/watch?v=nB84hmut8W0}{youtu.be/nB84hmut8W0}

