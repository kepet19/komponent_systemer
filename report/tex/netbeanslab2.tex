\section{NetBeansLab2}
Netbeans lab 2 skal man vise at netbeans module system kan loade og undeloade
moduler i runtime/kørslen af programmet. Der der allerede er lavet et modul
"SilentUpdate" som kan søger for at installere og afinstallér moduler i
runtime. Det modul smider man så som en dependecy inde i "application/pom.xml"
filen.

\subsection{Register komponent}
Denne her gang fjerner jeg det module jeg lavede i netbeans lab 1
"Astroid6Shapes" dependecy fra "application/pom.xml" for at vise den kan bliver
loadet ind med SilentUpdate.


\hfill \linebreak \hline 
\dirtree{%
.1 SilentUpdate.
.2 pom.xml.
.2 src.
.3 main.
.4 java.
.5 org.netbeans.modules.autoupdate.silentupdate.
.6 UpdateActivator.java.
.6 UpdateHandler.java.
.3 nbm.
.4 manifest.mf.
.3 resources.
.4 org.netbeans.modules.autoupdate.silentupdate.
.5 resources.
.6 Bundle.properties.
.6 layer.xml.
}
\hline \hfill \linebreak

I filen "Bundle.properties" skal man pege på ens update center. jeg har valgt at 
bruge "target" mappen i application mappen som "update site". 

Her under ses den bundle.properties. Linje 7 hvor man kan ændre hvor "netbeans
site" peger til.  Jeg har dog valgt at ændre navnet til kort et. da det ikke
ville værer pænt med full path til filen.

\begin{lstlisting}[caption={Bundle.properties}, label={lst:App}]
#Tue Mar 10 14:13:59 CET 2015
Services/AutoupdateType/org_netbeans_modules_autoupdate_silentupdate_update_center.instance=Sample Update Center
OpenIDE-Module-Display-Category=Infrastructure
OutputLogger.Grain=VERBOSE
OpenIDE-Module-Name=Silent Update
OpenIDE-Module-Short-Description=Silent Update of Application
org_netbeans_modules_autoupdate_silentupdate_update_center=file:///target/netbeans_site/updates.xml
OpenIDE-Module-Long-Description=A service installing updates of your NetBeans Platform Application with as few as possible user's interactions.
\end{lstlisting}

\subsection{Netbeans site}
Netbeans site modulerne består af information om hvad de vil have, hvad de giver
af service til andre og de indeholder også java byte code (jar filer).

\subsection{Komponentdet bliver loadet og undloadet via silent update}
Se video: \href{https://www.youtube.com/watch?v=H2Y0cPQ0fzc}{youtu.be/H2Y0cPQ0fzc}


