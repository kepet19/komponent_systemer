\section{SpringLab}
I denne opgave skal man porte \textbf{[JavaLab]}  til at være basseret på Spring
framworket.
Det har jeg gjort ved først at kigge på \textbf{[SpringAgeCalculator]} for at få
en ide om hvordan det virker.

Derefter åbenede jeg project fra \textbf{[JavaLab]} og fandt ud af hvilke moduler som skulle depency injectes via
spring frameworket.
Dem der skulle konventeres er:
\begin{itemize}
    \setlength\itemsep{-0.5em}
    \item \textbf{Asteroid} 
    \item \textbf{Collision} 
    \item \textbf{Enemy} 
    \item \textbf{Player} 
\end{itemize}

\subsection{Enemy modulet}%
\label{sub:enemy_module}


Jeg tager udgangspunkt I \textbf{Enemy}. 
Fil træet for enemy ser sådan ud. 

\hfill \linebreak \hline
\dirtree{%
.1 Enemy.
.2 src.
.3 main.
.4 java.
.5 dk.sdu.mmmi.cbse.enemysystem.
.6 Enemy.java.
.6 EnemyControlSystem.java.
.6 EnemyPlugin.java.
.3 resources.
.4 enemyBean.xml.
.2 pom.xml.
}
\hline \hfill \linebreak

Det vigtigste der har ændret sig i filen strukturen er der er kommet en ny
fil(\textbf{enemyBean.xml}) der beskriver et id på en bean og hvor man kan
finde den. Jeg har også slettet mappen \textbf{META-INF} der blev brugt til
Java's service loader. \\ enemyBean.xml kan ses i \textbf{\textbf{Listing
\ref{lst:enemyBean.xml}}}. Beans id'erne skal bruges i den bean hvor jeg samler
det hele (kommer senere).

\begin{lstlisting}[caption={enemyBean.xml}, label={lst:enemyBean.xml}, language=xml]
<?xml version="1.0" encoding="UTF-8"?>
<beans>
    <bean id="enemyControlSystem" class="dk.sdu.mmmi.cbse.enemysystem.EnemyControlSystem"/>
    <bean id="enemyPlugin" class="dk.sdu.mmmi.cbse.enemysystem.EnemyPlugin"/>

</beans>
\end{lstlisting}


\subsection{Core modulet}%
\label{sub:core_modulet}



I \textbf{Core} modulet skal der en ny dependency in i pom.xml. filen kan ses i
\textbf{\textbf{Listing \ref{lst:spring:core:pom.xml}}}. Der kommer jeg spring frameworket
dependency'en ind.


\begin{lstlisting}[caption={core:pom.xml}, label={lst:spring:core:pom.xml}, language=xml, firstnumber=92]
        <dependency>
            <groupId>org.springframework</groupId>
            <artifactId>spring-context</artifactId>
            <version>4.1.5.RELEASE</version>
        </dependency>
\end{lstlisting}

Derefter opretter jeg en ny fill der hedder \textbf{Beans.xml} den står så for
at beskrive hvordan bean ser ud i filen. Bean id'et er "core" og klassen er
Game. I Beans.xml filen importer jeg også de andre xml filer. På linje
6 i \textbf{Listing \ref{lst:Beans.xml}} kan man se at den tidligere beskrevet
enemyBean.xml bliver importeder her. \\
I den bean man er i gang med at beskrive(core) er der en property. Den property
har et navn som skal være tilgændelig som en sætter methode i Game klassen før
spring kan dependency injecte de andre klasser. Property'en har en liste med
"value-type" som siger at listen er af dette interface og pejer på interfacet
klassen.  klasserne som implementere det tidligere nævnte interface som man
gerne vil have dependency injectet skal man putte i listen. Listen består af
id'er på de andre beans. 


\hfill \linebreak
\hline
\dirtree{%
.1 Core.
.2 pom.xml.
.2 src.
.3 main.
.4 java.
.5 dk.sdu.mmmi.cbse.
.6 main.
.7 Game.java.
.7 Main.java.
.6 managers.
.7 GameInputProcessor.java.
.3 resources.
.4 Beans.xml.
}
\hline
\hfill \linebreak

\begin{lstlisting}[caption={Beans.xml}, label={lst:Beans.xml}, language=xml]
<?xml version="1.0" encoding="UTF-8"?>
<beans>
    <import resource="asteroidsBean.xml" />
    <import resource="collisionBean.xml" />
    <import resource="playerBean.xml" />
    <import resource="enemyBean.xml" />

    <bean id="core" class="dk.sdu.mmmi.cbse.main.Game">
        <property name="entityProcessors">
            <list value-type="dk.sdu.mmmi.cbse.common.services.IEntityProcessingService">
                <ref bean="playerControlSystem"/>
                <ref bean="asteroidControlSystem"/>
                <ref bean="asteroidSplitter"/>
                <ref bean="enemyControlSystem"/>

            </list>
        </property>
        <property name="postEntityProcessors">
            <list value-type="dk.sdu.mmmi.cbse.common.services.IPostEntityProcessingService">
                <ref bean="collider"/>
                <ref bean="asteroidPlugin"/>
            </list>
        </property>
        <property name="pluginServices">
            <list value-type="dk.sdu.mmmi.cbse.common.services.IGamePluginService">
                <ref bean="playerPlugin"/>
                <ref bean="asteroidPlugin"/>
                <ref bean="enemyPlugin"/>
            </list>
        </property>
    </bean>

</beans>
\end{lstlisting}

I \textbf{\textbf{Listing \ref{lst:spring:Game.java}}} ses der er lavet endnu en liste med
gamePlugin.

\begin{lstlisting}[caption={dk.sdu.mmmi.cbse.main.Game.java}, label={lst:spring:Game.java}, language=java, firstnumber=13]
public class Game
        implements ApplicationListener {

    private static OrthographicCamera cam;
    private ShapeRenderer sr;

    private final GameData gameData = new GameData();
    private List<IEntityProcessingService> entityProcessors = new ArrayList<>();
    private List<IPostEntityProcessingService> postEntityProcessors = new ArrayList<>();
    private List<IGamePluginService> gamePlugin = new ArrayList<>();
\end{lstlisting}

I \textbf{\textbf{Listing \ref{lst:spring:Game.java.2}}} ses der er lavet 3 methoder så spring kan
dependency injecte klasserne.

\begin{lstlisting}[caption={dk.sdu.mmmi.cbse.main.Game.java}, label={lst:spring:Game.java.2}, language=java, firstnumber=109]
    public void setPluginServices(List<IGamePluginService> gamePlugin) {
        this.gamePlugin = gamePlugin;
    }

    public void setPostEntityProcessors(List<IPostEntityProcessingService> 
            postEntityProcessors) {
        this.postEntityProcessors = postEntityProcessors;
    }

    public void setEntityProcessors(List<IEntityProcessingService> 
            entityProcessors) {
        this.entityProcessors = entityProcessors;
    }
}
\end{lstlisting}


For at oprette et game objecet/klasse med spring framworket skal man importere
spring framworket ClassPathXmlApplicationContext. Kalde konstrukteren på
ClassPathXmlApplicationContext med parameteret på den Bean.xml man har lavet.
Efter at have fået applicationContex fra ClassPathXmlApplicationContext kan man
få selve bean ud ved at kalde methoden \code{getBean()} med et parameter id.
Id'et er fra den bean man gerne vil have. Vi kunne godt tænke us den fra core
som er game klassen. Det kan ses i \textbf{Listing \ref{lst:spring:Main.java}} på linje
10. Der caster vi den så til et Game object, så den kan bruges normalt.
Spring frameworket har sørget for at gører de dele som er beskrevet i
\code{Beans.xml} og dependency inject de klasser som er beskrevet.


\begin{lstlisting}[caption={dk.sdu.mmmi.cbse.main.Main.java}, label={lst:spring:Main.java}, language=java, firstnumber=1]
package dk.sdu.mmmi.cbse.main;

import com.badlogic.gdx.backends.lwjgl.*;
import org.springframework.context.support.ClassPathXmlApplicationContext;

public class Main {

    public static void main(String[] args) {
        var context = new ClassPathXmlApplicationContext("Beans.xml");
        Game game = (Game) context.getBean("core");

        LwjglApplicationConfiguration cfg =
        new LwjglApplicationConfiguration();
        cfg.title = "Asteroids";
        cfg.width = 500;
        cfg.height = 400;
        cfg.useGL30 = false;
        cfg.resizable = false;

        new LwjglApplication(game, cfg);

    }

    }
\end{lstlisting}
