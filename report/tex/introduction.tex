\title{Assignment Portfolio 1}
\author{Kevin K. M. Petersen \and Eksamens nr: 480900}
\maketitle

\section{Introduktion}
Link til kildekode \href{https://github.com/kepet19/komponent_systemer}{github.com/kepet19/komponent\_systemer}

Formålet med opgaven er at komme igennem de forskellige komponent basseret
arkitektur. De forskellige komponent arkitektur er: 


\begin{itemize}
 \item whiteboard modellen (Den der er standard i java)
 \item netbeans module system
 \item OSGi {Declarative Services \& Bundle Context API}
\end{itemize}

\section{Javalab}
I denne opgave skulle man lave et ekstra komponent som kunne loades ind
automatisk uden at blive tilføjet manuelt som dependencies i java filerne.

\subsection{Register komponent}
først lave jeg en ny "Enemy" komponent som er baseret på player, da mange af
tingende er de samme. 

Derefter tilføjede jeg Enemy til rodens/root "./pom.xml" filen, hvor den kom
til at stå som et ekstra modul så maven vidste den skulle compiles med.
Derefter tilføjede jeg "Enemy" til dependencies inde i "Core/pom.xml". så java
kunne finde de interfaces som vi ledte efter.  En af de key items er at man
skal have en "resource" mappe med hvor er er en "META-INF" mappe og i den mappe
har man en fil med navnet på det interface man siger man implementere. inde i filen
siger man hvilke klasse der implementere dette interface.

Her har jeg en træ structur af hvordan filerne ser ud.

\hfill \linebreak \hline 
\dirtree{%
.1 Enemy.
.2 pom.xml.
.2 src.
.3 main.
.4 java.
.5 dk.sdu.mmmi.cbse.enemysystem.
.6 Enemy.java.
.6 EnemyControlSystem.java.
.6 EnemyPlugin.java.
.3 resources.
.4 META-INF.
.5 navent på det interface man siger man implementere.
.5 dk.sdu.mmmi.cbse.common.services.IEntityProcessingService.
.5 dk.sdu.mmmi.cbse.common.services.IGamePluginService.
}
\hline \hfill \linebreak

\begin{lstlisting}[caption={dk.sdu.mmmi.cbse.common.services.IEntityProcessingService}, label={lst:App}, language=java]
dk.sdu.mmmi.cbse.enemysystem.EnemyControlSystem
\end{lstlisting}
""
\begin{lstlisting}[caption={dk.sdu.mmmi.cbse.common.services.igamepluginservice}, label={lst:app}, language=java]
dk.sdu.mmmi.cbse.enemysystem.enemyplugin
\end{lstlisting}


\subsection{Hvordan systemet bruger den ny komponent}
Java søger søger i Core modulet efter dependencies der implementere nogle
besteme interfaces vi har valgt.




% ======{NetBeansLab1}==========================================================
\newpage
\section{NetBeansLab1}
I denne ogpave skal man lave et netbean module som kan loades ind via
"Netbeans module system". Der er allered givet example projekt, som jeg har
udvidet på.

\subsection{Register komponent}
For at registere et komponent skal man først lave et netbeans module som skal
lægges ind som dependenci inde i "application/pom.xml", så netbeans module
system ved at den skal loade modulet som der er blevet lavet.

\hfill \linebreak \hline 
\dirtree{%
.1 Astroid6Shapes.
.2 pom.xml.
.2 src.
.3 main.
.4 java.
.5 dk.sdu.mmmi.cbse.astroid6shapes.
.6 \textbf{AsteroidPlugin.java}.
.6 AsteroidProcessor.java.
.6 AsteroidSplitterImpl.java.
.3 nbm.
.4 manifest.mf.
.3 resources.
.4 dk.sdu.mmmi.cbse.astroid6shapes.
.5 Bundle.properties.
}
\hline \hfill \linebreak

I \textbf{Listing \ref{lst:AsteroidPlugin}} ses implementation på klassen
\textbf{AsteroidPlugin.java}. På linje 6 ses en decorator der fortæller netbeanas module 
system at dette er en klasse som provider IGamePluginService interfacet.

\begin{lstlisting}[caption={AsteroidPlugin.java}, label={lst:AsteroidPlugin}, language=java]
package dk.sdu.mmmi.cbse.astroid6shapes;

import dk.sdu.mmmi.cbse...*;
import org.openide.util.lookup.ServiceProvider;

@ServiceProvider(service = IGamePluginService.class)
public class AsteroidPlugin implements IGamePluginService{
	@Override
    public void start(GameData gameData, World world) {
        ...
    }

    @Override
    public void stop(GameData gameData, World world) {
		world.getEntities(Astroid6shapes.class).forEach(asteroid -> {
			world.removeEntity(asteroid);
		});
    }
}
\end{lstlisting}

\subsection{Netbeans module system lille og stor}
Netbeans module system har 2 forskellige systemer en slim uden noget grafisk,
kun til at loade netbeans modules med og en stor hvor man får netbeans grafiske
interface med.



% ======{NetBeansLab2}==========================================================
\newpage
\section{NetBeansLab2}
Netbeans lab 2 skal man vise at netbeans module system kan loade og undeloade
moduler i runtime/kørslen af programmet. Der der allerede er lavet et modul
"SilentUpdate" som kan søger for at installere og afinstallér moduler i
runtime. Det modul smider man så som en dependecy inde i "application/pom.xml"
filen.

\subsection{Register komponent}
Denne her gang fjerner jeg det module jeg lavede i netbeans lab 1
"Astroid6Shapes" dependecy fra "application/pom.xml" for at vise den kan bliver
loadet ind med SilentUpdate.


\hfill \linebreak \hline 
\dirtree{%
.1 SilentUpdate.
.2 pom.xml.
.2 src.
.3 main.
.4 java.
.5 org.netbeans.modules.autoupdate.silentupdate.
.6 UpdateActivator.java.
.6 UpdateHandler.java.
.3 nbm.
.4 manifest.mf.
.3 resources.
.4 org.netbeans.modules.autoupdate.silentupdate.
.5 resources.
.6 Bundle.properties.
.6 layer.xml.
}
\hline \hfill \linebreak

I filen "Bundle.properties" skal man pege på ens update center. jeg har valgt at 
bruge "target" mappen i application mappen som "update site". 

Her under ses den bundle.properties. Linje 7 hvor man kan ændre hvor "netbeans
site" peger til.  Jeg har dog valgt at ændre navnet til kort et. da det ikke
ville værer pænt med full path til filen.

\begin{lstlisting}[caption={Bundle.properties}, label={lst:App}]
#Tue Mar 10 14:13:59 CET 2015
Services/AutoupdateType/org_netbeans_modules_autoupdate_silentupdate_update_center.instance=Sample Update Center
OpenIDE-Module-Display-Category=Infrastructure
OutputLogger.Grain=VERBOSE
OpenIDE-Module-Name=Silent Update
OpenIDE-Module-Short-Description=Silent Update of Application
org_netbeans_modules_autoupdate_silentupdate_update_center=file:///target/netbeans_site/updates.xml
OpenIDE-Module-Long-Description=A service installing updates of your NetBeans Platform Application with as few as possible user's interactions.
\end{lstlisting}

\subsection{Netbeans site}
Netbeans site modulerne består af information om hvad de vil have, hvad de giver
af service til andre og de indeholder også java byte code (jar filer).

\subsection{Komponentdet bliver loadet og undloadet via silent update}
Se video: \href{https://www.youtube.com/watch?v=H2Y0cPQ0fzc}{youtu.be/H2Y0cPQ0fzc}


% ======{OSGiLab}===============================================================
\newpage
\section{OSGiLab}
I OSGI lab skal der laves 2 ny moduler, fordi der skal laves en med OSGi
Declarative Services og en med OSGi BundeContext API.

\subsection{OSGi Declarative Services}
I OSGi declarative services bruger vi en "osgi.bnd" til at fortælle OSGi hvilke
activationpolicy der skal værer. Den fortæller også noget om hvor den kan finde
noget om de service der bliver leveret fra modulet se \textbf{Listing
\ref{lst:osgi.bnd}}.

\hfill \linebreak \hline 
\dirtree{%
.1 Player.
.2 \textbf{osgi.bnd}.
.2 pom.xml.
.2 src.
.3 main.
.4 java.
.5 dk.sdu.mmmi.cbse.playersystem.
.6 Player.java.
.6 PlayerControlSystem.java.
.6 PlayerPlugin.java.
.3 resources.
.4 META-INF.
.5 entityprocessor.xml.
.5 \textbf{gameplugin.xml}.
}
\hline \hfill \linebreak

\begin{lstlisting}[caption={osgi.bnd}, label={lst:osgi.bnd}, language=java]
Bundle-SymbolicName: Player
Bundle-ActivationPolicy: lazy
Service-Component: META-INF/entityprocessor.xml, META-INF/gameplugin.xml
\end{lstlisting}

I \textbf{Listing \ref{lst:gameplugin.xml}} ses på linje 3 at det er
implementation klassen som den peger på og den implementere interfaces som er
listest på linje 5.

\begin{lstlisting}[caption={gameplugin.xml}, label={lst:gameplugin.xml}, language=xml]
<?xml version="1.0" encoding="UTF-8"?>
<scr:component xmlns:scr="http://www.osgi.org/xmlns/scr/v1.1.0" name="dk.sdu.mmmi.cbse.playersystem.plugin">
    <implementation class="dk.sdu.mmmi.cbse.playersystem.PlayerPlugin"/>
    <service>
        <provide interface="dk.sdu.mmmi.cbse.common.services.IGamePluginService"/>
    </service>
</scr:component>
\end{lstlisting}


% #### =========== bundle context API =========== ####
\newpage
\subsection{OSGI BundleContext API}

\hfill \linebreak \hline 
\dirtree{%
.1 Enemy2.
.2 \textbf{pom.xml}.
.2 src.
.3 main.
.4 java.
.5 dk.sdu.mmmi.cbse.osgienemy2.
.6 \textbf{Activator.java}.
.6 Enemy2.java.
.6 Enemy2Plugin.java.
.6 Enemy2Processor.java.
}
\hline \hfill \linebreak

\subsection{Register komponent}

I \textbf{Listing \ref{lst:pom-activator}} ses den pom.xml som peger på den
klasse som har implementeret bundle activator.

\begin{lstlisting}[caption={pom.xml}, label={lst:pom-activator}, language=xml]
    <build>
        <plugins>
            <plugin>
                <groupId>org.apache.felix</groupId>
                <artifactId>maven-bundle-plugin</artifactId>
                <extensions>true</extensions>
                <configuration>
                    <instructions>
                        <Bundle-Activator>dk.sdu.mmmi.cbse.osgienemy2.Activator</Bundle-Activator>
                    </instructions>
                </configuration>
            </plugin>
        </plugins>
    </build>
\end{lstlisting}


I \textbf{Listing \ref{lst:Activator.java}} ses den klasse som har implementeret
bundle activator.

\begin{lstlisting}[caption={Activator.java}, label={lst:Activator.java}, language=java]
package dk.sdu.mmmi.cbse.osgienemy2;

import dk.sdu.mmmi.cbse.common.services.IEntityProcessingService;
import dk.sdu.mmmi.cbse.common.services.IGamePluginService;
import org.osgi.framework.BundleActivator;
import org.osgi.framework.BundleContext;

public class Activator implements BundleActivator {

    @Override
    public void start(BundleContext context) throws Exception {

        context.registerService(IGamePluginService.class, new Enemy2Plugin(), null);
        context.registerService(IEntityProcessingService.class, new Enemy2Processor(), null);
    }

    public void stop(BundleContext context) throws Exception {
    }

}
\end{lstlisting}





\subsection{Komponentdet bliver loadet og undloadet via Apache gogo shell}
Se video \href{https://www.youtube.com/watch?v=nB84hmut8W0}{youtu.be/nB84hmut8W0}

